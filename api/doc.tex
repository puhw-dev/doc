\documentclass[a4paper]{article}
\usepackage[utf8]{inputenc}
\usepackage[T1]{fontenc}

\author{Michał Wielgus}
\title{Passive Uninterrupted Host Watcher - design document}
\date{2014-04-01}

\usepackage{fullpage}
\usepackage[english,polish]{babel}
\usepackage{graphicx}
\usepackage{hyperref}
\usepackage[hypcap]{caption}
\usepackage{perpage}
\usepackage{amsmath}
\usepackage{listings}
\usepackage{multicol}
\usepackage{multirow}

\renewcommand{\thefootnote}{\alph{footnote}}
\begin{document}
\pagenumbering{roman}
\maketitle
\tableofcontents
\listoffigures
\clearpage
\setcounter{page}{1} \pagenumbering{arabic}
%%%%%%%%%%%%%%%%%%%%%%%%%%%%%%%%%%%%%%%%%%%%%%%%%%%%%%%%%%%%%%%%%%%%%%%%%%%%%%%%
\section{Introduction}
\subsection{General}
\texttt{puhw} (Passive Uninterrupted Host Watcher) is a modular host monitoring system.
\section{Modules}
\subsection{Sensor}
Sensors periodically provide data to update the current state of relevant metrics.
\subsubsection{Configuration}
Sensors are configured locally with at least the following mandatory parameters (additional parameters may be required):
\begin{itemize}
	\item hostname
	\item sensorname
	\item monitor IP
	\item update frequency
	\item username \& password
\end{itemize}
\subsubsection{Communication}
A sensor can do only three things:
\begin{enumerate}
	\item Register with a monitor using hostname/sensorname pair, user credentials and a metrics list.
	\item Periodically send timestamped data for provided metrics.
	\item Receive a GTFO\footnote{Gracefuly Terminate Further Operation} request and die.
\end{enumerate}
All communication will be done via UDP.
The protocol is currently under development.
\subsection{Authentication, authorization}
A challenge-response scheme will be used for authentication.
Before registration and each metric update a challenge-response token is obtained from the assigned monitor; it is then appended to the prepared data, username, and password; hashed by a secure hash function and added to the message as a proof of sender's identity.

Given a sufficiently long token such a proof shall leave no doubt as to message's veracity, yet the user's credentials never have to be sent through an untrusted channel.

For the GTFO request this scheme is reversed, as it is the monitor's turn to prove its identity.
\subsubsection{Implementations}
Four sensor implementations will be provided:
\begin{description}
\item[SystemInfo] - 
	Provided metrics:
	\begin{itemize}
		\item hostname
		\item OS
		\item hardware info
		\item host description
	\end{itemize}
\item[SystemLoad] - 
	Provided metrics:
	\begin{itemize}
		\item free memory
		\item CPU utilization
		\item load average
		\item active threads
	\end{itemize}
\item[NetworkLoad] - 
	Provided metrics:
	\begin{itemize}
		\item TX/RX summary
		\item current bandwidth usage
	\end{itemize}
\item[Compound] - 
	Provided metrics:
	\begin{itemize}
		\item TX/RX summary
		\item current bandwidth usage
	\end{itemize}
	Additional parameters:
	\begin{itemize}
		\item Source Monitor
		\item Source Host
		\item Source Metric
		\item Window
	\end{itemize}
\end{description}
\subsection{Monitor}
Monitors gather, store and provide access to data sent by sensors.
They also handle authentication and allow for sensor termination.
\subsubsection{Configuration}
Monitors are configured locally with following mandatory parameters:
\begin{itemize}
	\item monitor name
	\item catalog IP
\end{itemize}
\subsubsection{Communication}
Monitors can do the following things:
\begin{itemize}
	\item register with a catalog
	\item update sensor info (\@ every sensor registration)
	\item sign a user in
	\item sign a user out
	\item add a user account
	\item provide a list of hosts
	\item provide a list of sensors for a selected host
	\item provide a list of metrics for a selected sensor
	\item provide a list of datapoints for a selected metric
\end{itemize}
\subsection{Catalog}
Catalog is a HTTP server that provides a public registry for monitors to populate. The up-to-date registry of monitors maintained by catalog allows browsing and searching.
\subsubsection{Configuration}
Catalog server can be configured via command line options. Execute catalog with -h option to see configuration possibilities.
\subsubsection{Communication}
Catalogs can do the following things:
\begin{itemize}
	\item provide a list of all monitors that have been registered
	\item register a catalog entry
	\item update a catalog entry
	\item delete a catalog entry
	\item search monitors by name
\end{itemize}
\subsection{Client}
Two clients are provided. More details will follow.
\subsubsection{Client 1 - WebGUI}
	WebGUI - general. User control!
\subsubsection{Client 2 - TUI}
	Top-like client. Exact specs TBD.
\clearpage
\section{API}
\subsection{Catalog}
\begin{itemize}
	\item \texttt{\{catalogURI\}/monitors}
	\begin{description}
		\item{GET}: returns list of registered monitors in following format:
			\begin{verbatim}
				[{
					"name": "monitor1", "ip": "10.0.0.1", "href": "{catalog-uri}/monitors/monitor1"
				},{
					"name": "monitor2", "ip": "10.0.0.2:33333", "href": "{catalog-uri}/monitors/monitor2"
				}]
			\end{verbatim}
		If there are no registered monitors, it returns empty list: [], and HTTP status code 200.
		\item{POST}: registers new monitor within this catalog. Requires input in following format:
			\begin{verbatim}
				{
					"name": "monitor1", "ip": "10.0.0.1:12345"
				}
			\end{verbatim}
			If monitor with given \textit{name} already exists returns HTTP status code 409.\\
			If message body is empty, or has wrong syntax returns HTTP status code 400.\\
			If monitor has been successfully registered, returns HTTP status code 201.
		\item{PUT}: n/a
		\item{DELETE}: n/a
	\end{description}
	\item \texttt{\{catalogURI\}/monitors/\{monitorID\}}
	\begin{description}
			\item{GET}: returns parameters of monitor with id \textit{\{monitor-id\}} in following format:
				\begin{verbatim}
					{
						"name": "monitor1", "ip": "10.0.0.1", "href": "{catalog-uri}/monitors/monitor1"
					}
			\end{verbatim}
			If there is no registered monitor with id \textit{\{monitor-id\}} in this catalog it returns HTTP status code 410.
			\item{POST}: Updates parameters of monitor with id \textit{\{monitor-id\}}. It requires input in following format:
				\begin{verbatim}
					{
						"ip": "10.0.0.1"
					}
				\end{verbatim}
				If there is no registered monitor with id \textit{\{monitor-id\}} in this catalog it returns HTTP status code 410.\\
				If message body is empty, or has wrong syntax returns HTTP status code 400.\\
				\textbf{Note:} To rename registered monitor delete it first, then register again under new (unique) name, see POST \{catalog-uri\}/monitors.
			\item{PUT}: n/a
			\item{DELETE}: deletes catalog entry for monitor with id  \textit{\{monitor-id\}}.
			If there is no registered monitor with id \textit{\{monitor-id\}} in this catalog, it returns HTTP status code 410.
		\end{description}
	\item \texttt{\{catalogURI\}/search/monitor-name=\{searched-name\}}
	\begin{description}
		\item{GET}: returns list of monitors which have parameter \textit{name} that matches regular expression: .*\{searched-name\}.*, i.e. for GET \{catalogURI\}/search/monitor-name=\textbf{ii}  matching names would be 'wf\textbf{ii}s', 'WF\textbf{iI}S'. Output has following format:
			\begin{verbatim}
				[{
				"name": "monitor1", "ip": "10.0.0.1:55555", "href": "{catalog-uri}/monitors/monitor1"
				},{
				"name": "monitor2", "ip": "10.0.0.2", "href": "{catalog-uri}/monitors/monitor1"
				}]
			\end{verbatim}
		If there are no results for the query, it returns empty list: [], and HTTP status code 200.
		\item{POST}: n/a
		\item{PUT}: n/a
		\item{DELETE}: n/a
	\end{description}
\end{itemize}
\subsection{Monitor}
\begin{itemize}
	\item \texttt{\{monitorURI\}/hosts/}
	\begin{description}
		\item{GET}: get list of hosts serviced by this monitor
		\item{POST}: n/a
		\item{PUT}: n/a
		\item{DELETE}: n/a
	\end{description}
	\begin{verbatim}
	{
		"name": "monitor1", href: "http://10.0.0.1/hosts/",
		[{
			"hostname": "box", "ip" = "10.0.1.128", "href": "http://10.0.0.1/hosts/box"
		},{
			"hostname": "xob", "ip" = "10.0.1.129", "href": "http://10.0.0.1/hosts/xob"
		}]
	}
	\end{verbatim}
	\item \texttt{\{monitorURI\}/hosts/\{hostname\}/sensors/}
	\begin{description}
		\item{GET}: get list of sensors for a particular host
		\item{POST}: n/a
		\item{PUT}: n/a
		\item{DELETE}: n/a
	\end{description}
	\begin{verbatim}
	{
		"hostname": "box", "ip" = "10.0.1.128", "href": "http://10.0.0.1/hosts/box",
		"sensors": [{
			"sensorname": "sensor1", "owner": "user1", "rpm": "10",
			"href": "http://10.0.0.1/hosts/box/sensors/sensor1"
		},{
			"sensorname": "sensor2", "owner": "user1", "rpm": "10",
			"href": "http://10.0.0.1/hosts/box/sensors/sensor2"
		}]
	}
	\end{verbatim}
	\item \texttt{\{monitorURI\}/hosts/\{hostname\}/sensors/\{sensorname\}}
	\begin{description}
		\item{GET}: get info for a particular sensor
		\item{POST}: n/a
		\item{PUT}: n/a
		\item{DELETE}: send a GTFO request to a particular sensor
	\end{description}
	\begin{verbatim}
	{
		"sensorname": "sensor1",
		"hostname": "box",
		"href": "http://10.0.0.1/hosts/box/sensors/sensor1",
		"owner": "user1", "rpm": "10"
	}
	\end{verbatim}
	\item \texttt{\{monitorURI\}/hosts/\{hostname\}/sensors/\{sensorname\}/metrics/}
	\begin{description}
		\item{GET}: get list of metrics for a particular sensor
		\item{POST}: n/a
		\item{PUT}: n/a
		\item{DELETE}: n/a
	\end{description}
	\begin{verbatim}
	{
		"sensorname": "sensor1",
		"hostname": "box",
		"owner": "user1", "rpm": "10",
		"href": "http://10.0.0.1/hosts/box/sensors/sensor1",
		"metrics": [{
			"name": "metric1", "href": "http://10.0.0.1/hosts/box/sensors/sensor1/metrics/metric1"
		},{
			"name": "metric2", "href": "http://10.0.0.1/hosts/box/sensors/sensor1/metrics/metric1"
		}]
	}
	\end{verbatim}
	\item \texttt{\{monitorURI\}/hosts/\{hostname\}/sensors/\{sensorname\}/metrics/\{metricname\}}
	\begin{description}
		\item{GET}: get info for a particular metric
		\item{POST}: n/a
		\item{PUT}: n/a
		\item{DELETE}: n/a
	\end{description}
	\begin{verbatim}
	{
		"sensorname": "sensor1",
		"hostname": "box",
		"owner": "user1", "rpm": "10",
		"href": "http://10.0.0.1/hosts/box/sensors/sensor1",
		"metrics": [{
			"name": "metric1", "href": "http://10.0.0.1/hosts/box/sensors/sensor1/metrics/metric1"
		},{
			"name": "metric2", "href": "http://10.0.0.1/hosts/box/sensors/sensor1/metrics/metric1"
		}]
	}
	\end{verbatim}
	\item \texttt{\{monitorURI\}/hosts/\{hostname\}/sensors/\{sensorname\}/metrics/\{metric1\};\{metric2\}/data[\&n=20]}
	\begin{description}
		\item{GET}: get $n$ (max $100$, default $20$) timestamped datapoints for the requested metric
		\item{POST}: n/a
		\item{PUT}: n/a
		\item{DELETE}: n/a
	\end{description}
	\begin{verbatim}
	{
		"sensorname": "sensor1",
		"hostname": "box",
		"owner": "user1", "rpm": "10",
		"href": "http://10.0.0.1/hosts/box/sensors/sensor1/metrics/",
		"metrics": [{
			"name": "metric1", "href": "http://10.0.0.1/hosts/box/sensors/sensor1/metrics/metric1",
			"data": [("2014-04-01T01:01:01", 1), ("2014-04-01T01:01:01", 2),
			("2014-04-01T01:01:01", 3), ...]
		},{
			"name": "metric2", "href": "http://10.0.0.1/hosts/box/sensors/sensor1/metrics/metric1",
			"data": [("2014-04-01T01:01:01", 1), ("2014-04-01T01:01:01", 2),
			("2014-04-01T01:01:01", 3), ...]
		}]
	}
	\end{verbatim}
\end{itemize}
\clearpage
\section{Subteams}
\subsection{Catalog}
\begin{itemize}
\item Maciej Siczek
\end{itemize}
\subsection{Monitor}
\begin{itemize}
\item Remigiusz Rohulko
\item Marek Nędza
\item Krzysztof Żygłowicz
\end{itemize}
\subsection{Sensors - monitor communication}
\begin{itemize}
\item Dominik Szerszeń
\end{itemize}
\subsection{Sensors}
\begin{itemize}
\item Przemysław Plutecki
\item Patryk Konopka
\end{itemize}
\subsection{Client 1}
\begin{itemize}
\item Michał Wąsek
\item Paweł Piecyk
\end{itemize}
\subsection{Client 2}
\begin{itemize}
\item Paweł Zadrożniak
\end{itemize}
\subsection{Architecture, PR}
\begin{itemize}
\item Michał Wielgus
\end{itemize}
\section{Talk}
\begin{itemize}
\item mockup
\item udp
\item register/login
\item auth
\end{itemize}
\end{document}
